\documentclass{acm_proc_article-sp}

\begin{document}

\title{Minimizing Query Transfer in Distributed IR\titlenote{A summary of previous papers and a suggestion of new ideas}}

\numberofauthors{1} 
\author{
\alignauthor
Edward Hills\\\affaddr{University of Otago}\\
       \affaddr{Dunedin, New Zealand}\\
       \email{ehills@cs.otago.ac.nz}}
\date{18 May 2012}

\maketitle
\begin{abstract}
Query processing and optimization in a Distributed IR architecture is fundamental to todays IR needs. Due to the high usage of web searches in a distributed environment as well as P2P Distributed IR which is gaining in popularity very quickly, there is a strong need to be able to process a given query, work out its merged postings list and return the desired documents back to the user in a quick and successive fashion.

This paper discusses techniques previously established for minimizing the overhead and pain when trying to merge the postings list that each node in a distributed environment produces.
\end{abstract}

\section{Introduction}

Blah Blah Blah, talk about the main points that each of the papers produce blah blah blah. then talk about a couple of questions youll bring up.

\section{Query Optimization}

Briefly touch on the idea of optimizing this

\subsection{Research Questions}
\subsubsection{Paper 1}
Briefly talk about the research questions for this paper

\subsubsection{Paper 2}
Briefly talk about the research questions for this paper

\subsubsection{Paper 3}
Briefly talk about the research questions for this paper

\subsection{Main Contributions}
Briefly touch on the main contributions for all papers
\subsubsection{Paper 1}
Describe the main contributions for this paper.

\subsubsection{Paper 2}
Describe the main contributions for this paper.

\subsubsection{Paper 3}
Describe the main contributions for this paper.

\subsection{Relationship}
Talk about how all three papers are related to each other

\section{Future Work}
I came up with two ideas in this field which are roughly to do with this... blah blah blah

\subsection{Question 1}
Well the first question i wanted answered is, can we do this... blah blah blah. the best way to do this would be combine this this and that and then it may be possible to imporve performance or not blah blah blah

\subsection{Question 2}
Well the first question i wanted answered is, can we do this... blah blah blah. the best way to do this would be combine this this and that and then it may be possible to imporve performance or not blah blah blah

\section{Conclusions}
You can see that this paper has described a multitude of different ways in which we can minimize the cost of transfering or merging the queries in a distributed IR environment. by taking points from paper 1 and paper 2 they can be combined and blah blah blah. Paper 3 talks aboue the architecture and we can see that this is important because of blah.

I came up with my own 2 points and found that blah blah blah

\bibliographystyle{abbrv}
\bibliography{asgn2bib} 

\subsection{References}
Generated by bibtex from your ~.bib file.  Run latex,
then bibtex, then latex twice (to resolve references)
to create the ~.bbl file.  Insert that ~.bbl file into
the .tex source file and comment out
the command \texttt{{\char'134}thebibliography}.
\balancecolumns
% That's all folks!
\end{document}
