\documentclass{acm_proc_article-sp}

\begin{document}

\title{Minimizing Posting List Traversal\titlenote{A summary of previous papers and a suggestion of new ideas}}

\numberofauthors{1} 
\author{
\alignauthor
Edward Hills\\\affaddr{University of Otago}\\
       \affaddr{Dunedin, New Zealand}\\
       \email{ehills@cs.otago.ac.nz}}
\date{18 May 2012}

\maketitle
\begin{abstract}

Information Retrieval is primarily concerned with searching through a document collection given a query, and returning a set of documents which could be relevant to the users request. 

This task usually requires searching through the entire document collection to find which documents may be relevant. In large document collections this can be time consuming and costly. This paper looks at ways to minimize the number of documents that are examined but still being effective in returning the documents most likely to be relevant.

\end{abstract}

\section{Introduction}

Searching through document collections which are large can mean when it comes to query time that all documents that are indexed by one or more of the search terms will be examined. When you have a large document collection such as the TREC terabyte collection, searching every document that is indexed by even a relatively rare term can mean searching through thousands or even millions of documents.

By reducing the amount of documents searched by a particular query you do run the risk of limiting the amount of documents returned which are relevant to the user. Keeping the accuracy of your list returned is a top priority when searching through a minimal amount of documents. 

There are a number of techniques in which the document list traversal can be minimized while still maintaining high accuracy. This paper looks at three techniques, \emph{Index ordering by query-independent measures, Ferguson, Smeaton} which is looking at documents on the web in particular, \emph{The Nearest Neighbour Problem In Information Retrieval, van Rijsbergen, Smeaton} and \emph{Filtered Document Retrieval with Frequency-Sorted Indexes, Persin, Zobel, Sacks-Davis}. 

I will examine document minimization further and discuss the main research questions and contributions of each paper in 2., talk about how these techniques are linked in 3., talk about two new proposals I have come up with in 4., and discuss my final findings in the Conclusion.

\section{Document Minimization}

Briefly touch on the idea of optimizing this

\subsection{Research Questions}
\subsubsection{Paper 1}
Briefly talk about the research questions for this paper

\subsubsection{Paper 2}
Briefly talk about the research questions for this paper

\subsubsection{Paper 3}
Briefly talk about the research questions for this paper

\subsection{Main Contributions}
Briefly touch on the main contributions for all papers
\subsubsection{Paper 1}
Describe the main contributions for this paper.

\subsubsection{Paper 2}
Describe the main contributions for this paper.

\subsubsection{Paper 3}
Describe the main contributions for this paper.

\subsection{Relationship}
Talk about how all three papers are related to each other

\section{Future Work}
I came up with two ideas in this field which are roughly to do with this... blah blah blah

\subsection{Question 1}
Well the first question i wanted answered is, can we do this... blah blah blah. the best way to do this would be combine this this and that and then it may be possible to imporve performance or not blah blah blah

\subsection{Question 2}
Well the first question i wanted answered is, can we do this... blah blah blah. the best way to do this would be combine this this and that and then it may be possible to imporve performance or not blah blah blah

\section{Conclusions}
You can see that this paper has described a multitude of different ways in which we can minimize the cost of transfering or merging the queries in a distributed IR environment. by taking points from paper 1 and paper 2 they can be combined and blah blah blah. Paper 3 talks aboue the architecture and we can see that this is important because of blah.

I came up with my own 2 points and found that blah blah blah

\bibliographystyle{abbrv}
\bibliography{asgn2bib} 

\subsection{References}
Generated by bibtex from your ~.bib file.  Run latex,
then bibtex, then latex twice (to resolve references)
to create the ~.bbl file.  Insert that ~.bbl file into
the .tex source file and comment out
the command \texttt{{\char'134}thebibliography}.
\balancecolumns
% That's all folks!
\end{document}
